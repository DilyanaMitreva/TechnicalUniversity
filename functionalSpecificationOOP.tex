\documentclass[a4paper,fleqn,12pt]{article}

\usepackage[utf8]{inputenc}
\usepackage[bulgarian]{babel}
\usepackage{amsmath}
\usepackage{amssymb}
\usepackage{booktabs}
\usepackage{fancyhdr}
\usepackage{amsthm}
\usepackage{graphicx}

\title{
  Функционална спецификация \\
  \large "Приложение за работа с графични фигури"}

\author{Кристиян Кръчмаров}

\pagestyle{fancy}
\fancyhf{}
%\lhead{\rightmark}
\rhead{\thepage}
\cfoot{}
\renewcommand{\headrulewidth}{0pt}

\begin{document}

\pagenumbering{gobble}
\maketitle
\newpage

\tableofcontents
\newpage

\newpage
\pagenumbering{arabic}

\section{Структура на проекта}
Проекта ще реализира следната йерархия: \\
Базов клас Shape с три наследника: Circle, EquilateralTriangle, Rectangle. 

\subsection{Shape}
Класът Shape ще съдържа в себе си обща информация за всички фигури наследници, като началните кординати и цветовете за запълване и чертане, както и метод за визуализиране на фигурата. 
В класът ще има абстрактно свойство (property), отговарящо за площта на фигурата, абстрактни методи за проверка на дали точка попада във фигура и дали фигурата се пресича с правоъгълник.

\subsection{Circle}
Класът Circle ще реализира функционалност за фигурата кръг.
Ще притежава радиус и център на кръга. 

\subsection{EquilateralTriangle}
Класът EquilateralTriangle ще реализира функционалност за фигурата равностранен триъгълник. 
Той ще притежава страна и масив от 3 точки, които са върховете на триъгълника.

\subsection{Rectangle}
Класът Rectangle ще реализира функционалност за фигурата правоъгълник. 
Той ще притежава ширина и дължина на правоъгълника. 

\newpage

\section{Графичен интерфейс}
Проекта ще реализира графичен интерфейс (GUI), посредством Windows Forms с три основни прозореца: Прозорец на който да се визуализират фигурите, прозорец за въвеждане и промяна на информацията за съответната фигура и прозорец с допълнителна функционалност свързана с фигурите, създадени от потребителя. 

\subsection{Прозорец за визуализация (сцена)}
Този прозорец ще служи за визуализацията на фигурите, които биват създавани от потребителя. 
Ще се реализират следните взаимодействия между потребителя и програмата: 
\begin{itemize}
\item Добавяна на нова фигура с натискането на десния бутон на мишката. 
\item С натискане на левия бутон на мишката да се избира фигура.
\item С влачене на мишката с натиснат ляв бутон на мишката да избира всички фигури в правоъгълника, генериран от влаченето. 
\item С двойно натискане на левия бутон върху фигура, ще се предоставя възможност за промяна на данните на избрана фигура.
\item С натискане на клавиша Delete да се изтриват всички избранни фигури. 
\end{itemize}
Прозореца ще притежава няколко бутона, разположени в долната част.
Бутоните ще бъдат със следните функционалностти: 
\begin{itemize}
\item Бутон за изтриване на избрани фигури
\item Бутон за отваряне на допълнителната функционалност
\item Бутони за избиране на всички фигури, от даден тип. 
\end{itemize}
Ще бъде използвана библиотеката System.Graphics за визуализация на фигурите, посочени по горе.
Цветовете на фигурите ще бъдат избирани на случаен принцип. 

\subsection{Прозорец за данни}
Този прозорец ще отговаря за обработка на данни, създаване и промяна на фигури. 
Прозореца ще има реализирана валидация за данните, въведени от потребителя.
Ше се валидира дали въведените данни в текстовите полета са числа, по големи от нула и дали всичката информация е валидна при натискане на бутона за създаване на фигура. 
Прозореца ще дава възможност за избор на цветове за запълване и чертане. 
Ако потребителя не избере един от цветовете, ще бъде сложен случаен цвят, а ако не избере нито един, тогава ще бъдат сложен един цвят за чертане и изсветлен за запълване на фигурата. 

\subsection{Прозорец за допълнителна информация}
Toзи прозорец ще предостави на потребителя допълнителна информация свързана с фигурите, които са на сцената.  

\section{Допълнителна функционалност}
Йерархията на проекта ще бъде реализирана в отделна библиотека от визуалната част. 

\subsection{Съхранение}
Проекта ще има възможността за съхраняване на информацията, въведена от потребителя във текстов файл. 

\subsection{Функционалност, свързана с фигурите}
Гореспоменатата допълнителна функционалност ще реализира следните взаимодействия с фигурите.
\begin{itemize}
\item Колко е сумарното лице на всички фигури. 
\item Колко е сумарното лице на всички фигури от всеки тип.
\item Колко е най голямото и най малкото лице от всички фигури.
\item Колко е най голямото и най малкото лице от тип фигура.
\item Колко от площта на екрана не е заета от фигури. 
\end{itemize}
Тя ще бъде реализирана с Language Integrated Query (LINQ).

\end{document}