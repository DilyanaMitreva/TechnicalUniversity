\documentclass[a4paper,fleqn,12pt]{article}

\usepackage[utf8]{inputenc}
\usepackage[bulgarian]{babel}
\usepackage{amsmath}
\usepackage{amssymb}
\usepackage{booktabs}
\usepackage{fancyhdr}
\usepackage{amsthm}
\usepackage{graphicx}

\title{
  Функционална спецификация \\
  \large "Приложение за работа с графични фигури"}

\author{Кристиян Кръчмаров}

\pagestyle{fancy}
\fancyhf{}
\lhead{\rightmark}
\rhead{\thepage}
\cfoot{}
\renewcommand{\headrulewidth}{0pt}

\begin{document}

\pagenumbering{gobble}
\maketitle

\newpage
\pagenumbering{arabic}

\section{Етап 1}
Проекта ще реализира следната йерархия: \\
Един Базов клас Shape с три наследника: Circle, EquilateralTriangle, Rectangle. 
Всяка фигура ще притежава начални кординати.
Също така ще се съдържат размери за съответните фигури и методи свързани с различната функционалност на проекта. 
Фигурите ще се съхраняват в една колекция.
Ще има използване на различни модификатори за достъп (access modifiers).
Началните кординатите и размерите на фигурите ще бъдат представени със свойства (properties).

\section{Етап 2}
Проекта ще реализира графичен интерфейс (GUI), посредством Windows Forms с три основни прозореца: Прозорец на който да се визуализират фигурите, прозорец за въвеждане на информацията за съответната фигура и прозорец с допълнителна функционалност свързана с фигурите, създадени от потребителя. 
Ще бъде използвана библиотеката System.Graphics за визуализация на фигурите посочени по горе. \\
Ще се реализират следните взаимодействия между потребителя и програмата: 
\begin{itemize}
\item Добавяна на нова фигура с натискането на десния бутон на мишката. 
\item С натискане на левия бутон на мишката да се избира фигура.
\item С влачене на мишката с натиснат ляв бутон на мишката да избира всички фигури в правоъгълника, генериран от влаченето. 
\item С двойно натискане на ляв бутон върху да се предоставя възможност за промяна на данните на избрана фигура.
\item С натискане на клавиша Delete или чрез бутона Delete на екрана да се изтриват всички избранни фигури. 
\end{itemize}

\section{Етап 3}
Проекта ще има възможността за съхраняване на информацията, въведена от потребителя. 
Йерархията на проекта ще бъде реализирана в отделна библиотека от визуалната част. \\
Гореспоменатата допълнителна функционалност ще реализира следните взаимодействия с фигурите.
\begin{itemize}
\item Колко е сумарното лице на всички фигури. 
\item Колко е най голямото лице от всички фигури.
\item Колко от площта на екрана не е заета от фигури. 
\item Колко е сумарното лице на всички фигури от всеки тип. 
\end{itemize}
Тя ще бъде реализирана с Language Integrated Query (LINQ).



\end{document}